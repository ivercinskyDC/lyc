\documentclass[]{article}

%opening
\title{}
\author{}

\begin{document}

\maketitle

\begin{abstract}

\end{abstract}

\section{Ejercicio 2}

\begin{flushleft}
Sea $f_{1}(x,y) = x + y$\linebreak
\\Vamos a aplicar recursicion primitiva. 
\\Queremos que $h(x,y) = f_{1}(x,y) = x + y$
\\Es claro que necesitamos $n=1$
\\Luego $h(x,0) = x + 0 = x = id(x) \Rightarrow f(x) = id(x)$
\\Y, $h(x,t+1) = x+(t+1) = (x+t)+1 = s(x+t) = s(h(x,t)) = g(h(x,t),x,t)$
\\Entonces $g(x,y,z) = s(x)$
\\Luego, $g$ la podemos obtener por composicion $g(x,y,z)=s(u^{3}_{1}(x,y,z))$, con $f = s$ y $g_{1} = u^{3}_{1}$

\end{flushleft}

\begin{flushleft}
	Sea $f_{2}(x,y) = x * y$\linebreak
	\\Vamos a aplicar recursicion primitiva. 
	\\Queremos que $h(x,y) = f_{2}(x,y) = x * y$
	\\Es claro que necesitamos $n=1$
	\\Luego $h(x,0) = x * 0 = 0 = 0 \Rightarrow f(x) = 0$
	\\Y, $h(x,t+1) = x*(t+1) = x*t + x = h(x,t) + x = f_{1}(h(x,t),x)$
	\\Entonces 	$f_{1}(h(x,t),x) = g(h(x,t),x,t)$
	\\Entonces $g(x,y,z) = f_{1}(x,y)$
	\\Luego, $g$ la podemos obtener por composicion $g(x,y,z)=f_{1}(u^{3}_{1}(x,y,z),u^{3}_{2}(x,y,z))$, con $f = s$, $g_{1} = u^{3}_{1}$ y $g_{2} = u^{3}_{2}$
	
\end{flushleft}






\end{document}
